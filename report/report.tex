\documentclass[english]{article}

\usepackage{biblatex}
\addbibresource{bib.bib}

\title{Brain Image-Derived Phenotypes Regression Using Convolutional Neural Network}

\author{Bramantyo Ibrahim Supriyatno}

\date{March 2022}

\begin{document}
    \maketitle
    \begin{abstract}
        The use of convolutional neural networks (CNN) for computer vision problems has been significantly successful. 
        However, there is still a debate whether a complex non-linear model can outperform linear models for brain imaging data. 
        Knowing how each component of CNN affects its performance would be beneficial in order to address this debate. 
        In this project, we examine how each component of CNN contributes to the overall performance of the model in predicting brain image-derived phenotypes(IDPs). 
        We found that kernel size, strides number and deeper network have profound effects on the performance especially in gray-white contrast related phenotypes. 
    \end{abstract}
    

    \section*{Introduction}
    
    \section*{Method}
    lorem

    \section*{Results}
    lorem
    \section*{Conclussions and Summary}

    \section*{References}

    \printbibliography
\end{document}